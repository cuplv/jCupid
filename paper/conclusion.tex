Timing side-channels are a great concern when developing 
secure programs. With \jcupid, we hope to aid developers in writing
code that is timing side-channel free. In order to find the side
channels, \jcupid 
tracks the sequence of bytecodes that are executed for multiple inputs
of the same size. Once two inputs are found that cause a different
sequence of bytecodes to be executed, \jcupid can inform the developer
of the line that caused the
divergence. We show that in this way \jcupid can detect side channel
bugs using a series of case studies (and we show that \jcupid does not
detect bugs in these corrected versions).

This tool is intended to be used by developers to test very
specific aspects of their code. Running \jcupid on even moderately sized
projects, say an authentication server, will most
certainly find different bytecodes executed based on input (even of
the same length). \jcupid is intended as a more precise tool. Instead of 
analyzing the whole server, to continue the analogy, the tool is
intended to analyze whether the modular exponentiation function
executes different bytecodes for inputs of the same size (and thus for
instance reveals information about bits in the secret key). 
