

We evaluated jCupid using a set of simple programs that commonly contain timing
side channels in their implementations.


\paragraph{Password checking}
The na\"{\i}ve method of checking whether a provided password is correct is to iterate
over the provided input, comparing each character to its expected value, and
returning failure on the first discovered difference. This results in a side
channel that leaks the number of characters that an attacker has guessed
correctly, allowing them to perform an adaptive attack. The attacker can guess
each character independently, and only move on to the next character when they
have found the correct character in the current position. While we call this a
password checking test case, this pattern also occurs in any string comparison
that should be done in constant time, such as when checking if CSRF or
authentication tokens are valid.

\paragraph{Lucky-13 attack}
We implemented a toy Java version of the padding and MAC check employed in TLS
decryption. Our implementation was vulnerable to the Lucky 13 attack, where
an attacker modifies ciphertext, and can distinguish between the corresponding
plaintext having valid padding or not. This capability allows an attacker to
perform a Vaudenay attack and iteratively leak the plaintext.

\paragraph{SumBytes}
We created a program that reads a single byte to determine a length, then reads
that many bytes and hashes them. Here, we are mimicking a simple serialized
network protocol with a length encoded at the beginning of the message. The side
channel present in this code is subtle: there are no if statements or obvious
branches in the code. Rather, the number of bytes read determines the number of
compression function iterations occur internal to our hash function.

\paragraph{Multiplexer}
Finally, we implemented a selection function that takes three inputs: a, b, and
a selector. If the selector is true, our function returns the first input,
otherwise it returns the second input. This is easily accomplished by an if
statement, however, such an implementation may leak information due to instruction cache timing or
memory accesses.



For each of these programs, we allowed them to take a single input as a string
on standard input. Then, we allowed jCupid to fuzz these programs with random
inputs, and look for potential differences in the bytecodes executed. In each
example, we found the intended side-channel, and jCupid was able to correctly
determine the line of code responsible.

We then corrected the problems identified by jCupid, and reran our tool to
verify we had removed the offending side-channels. To our surprise, jCupid
identified additional problems in our ``corrected'' programs. For example, in
the password checking program, we used a temporary value, and updated it at each
iteration with $ good \&= (input[i] == expected[i]) $. However, this resulted in
a side-channel when computing the $ == $ operator, as Java implemented this as a
branch in the bytecode. We fixed this by switching to using exclusive-or to
compare the values, and jCupid did not detect any additional side-channels. This
further illustrates the difficulty of removing all potential side-channels from
code: even seemingly branch-free programs, written with the intention of not
having side-channels, can contain them.


As jCupid is a dynamic analysis tool, it must run many instances of the program
in order to find side-channel behavior. We evaluated jCupid's overhead by
measuring how quickly it can run basic Java programs, and compared it to running
those same programs outside of the jCupid environment. Particularly for a
developer dynamic analysis tool, our results are encouraging: per run, jCupid
only adds approximately X\% of overhead. This could allow even rare
side-channels to be detected with na\"{\i}ve fuzzing of nightly builds over unit
tests.

\paragraph{Timing experimentation}
In order to test the efficiency of jCupid, we ran a number of experiments to
determine how much overhead jCupid has, and where the overhead comes from.
With the above test programs we ran number of tests based on the modifications
we had made. We ran each test program 10 times with OpenJDK, but no flags added
to get a base value, then we ran the programs with our tool, telling it to use
at most 10 iterations, and ensured that our tool would not catch any differences
with the inputs. Then we ran the programs with a modified version of our tool 
which simply calls OpenJDK, with no flags -- this allows us to test how much 
overhead the tool itself has. Finally we run the tool on a modified OpenJDK such
that the hash function does no work (meaning nothing will be flagged) -- this 
tells us how much time our hash function is taking. Each of the above tests are 
run 5 times.

Averaging the time results across the different programs provides the following
results: just calling OpenJDK takes around 3.4 seconds, jCupid averages 6.2 seconds
per call to OpenJDK. This is an overhead of 2.8 seconds. The following two results
will help us determine where this overhead comes from. With the modified jCupid tool
that just calls OpenJDK with no flags takes 4.6 seconds, and finally calling jCupid
that calls a modified OpenJDK that does not do hashing takes 5.6 seconds.

The above results tell us that of the 2.8 seconds of overhead 1.2 seconds of overhead
come from the tool itself, and 1.0 second of overhead comes from the flags that
we use in jCupid. Thus the remaining 0.6 seconds of overhead come from the hash
function itself.
\paragraph
