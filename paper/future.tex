
There are many avenues for future work on this project. In particular, 
we plan to use, and potentially adapt, a concolic tester, in order to
cover all control-flow paths in a program. Our initial experiments
show that jCute~\cite{conf/cav/SenA06} was much faster
at finding inputs of interest than the random fuzzer that we
implemented (as would be expected). We would need to adapt the
concolic tester as not only would we want this tester to
be able to find all paths, but also find paths that are reachable under
given constraints, such as input length. 

Another interesting avenue is to refine this tool to examine the types
of differences we see in bytecode execution. In particular, it may be
that certain programs are written so that there are no timing
difference, but there are bytecode execution differences. As the goal
of the tool is to find timing side channels, this would lead to
false positives. Thus it would be interesting to 
collect bytecodes into equivalence classes by time and allowing
differences in execution within these classes. 

This project was written for Java programs, it would be a noteworthy
extension to continue this project in C to let a greater audience
access to this tool. 
