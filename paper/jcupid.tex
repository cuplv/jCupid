
\jcupid consists of a python script that wraps a modified version of OpenJDK.
Our python wrapper runs a specified program multiple times through our modified
OpenJDK, looking for two inputs that provide different bytecodes executed. We
have several optimizations to make this performant, which we detail in this
section.

\subsection{OpenJDK}
\label{sec:OpenJDK}

While OpenJDK already provides a feature for printing out executed bytecodes in
debug mode (with the \texttt{-XX:+TraceBytecodes} flag), we found that using
this feature for our tool proved to be significantly too slow. In addition,
this prints out a large number of bytecodes used in the initialization and
startup of the program. For example, a simple ``Hello World'' program executes
1.18 million bytecodes, while only 6,054 of those bytecodes are actually within
the \texttt{main} method. In addition, even just running this simple program
takes 8.6 seconds to execute using the \texttt{-XX:+TraceBytecodes} flag.
% 1,179,829 bytecodes executed total

Rather than using the \texttt{-XX:+TraceBytecodes} option and inspecting the
output to determine the bytecodes executed for a given program run, we modified
OpenJDK and added a new feature that hashed the bytecodes as they were executed,
and then printed out the final hash after the program terminated. As we do not
need a cryptographic hash for this purpose, we used the lightweight djb2 hash
function~\cite{djb2Hash}.

In addition to hashing, we also added a feature to allow the user
to specify the class and method for when this hashing should start and end, so
that extraneous bytecodes executed during program initialization can be ignored.
This feature is also available during the use of the
\texttt{-XX:+TraceBytecodes} option, limiting the bytecodes that are printed to
the method specified by the user. With these two features, a simple ``Hello
World'' program can be run in about 0.8 seconds, ten times faster than using
the \texttt{-XX:+TraceBytecodes} flag.

With this feature, our python wrapper can simply generate inputs, and receive
the hash of the bytecodes executed in the relevant class/method for that input.
Once the python script determines two inputs that produce different hashes, it
knows the execution has diverged. The script then follows up by using the
slower \texttt{-XX:+TraceBytecodes} flag on both instances of the program, and
compares their outputs to detect the deviation. 

%To help in this we modified OpenJDK to allow for new flags: \texttt{-hashClass}, \texttt{-hashMethod}, \texttt{-traceClass}, and \texttt{-traceMethod}. Each takes an argument, a class name or method name respectively. The \texttt{-hashClass} and \texttt{-hashMethod} flags are used together as are the trace variants. Their arguments dictate the class and method that we are interested in. For example in Listing~\ref{lst:ex} we would be interested in the class \texttt{SumRandomBytes} and the method \texttt{main}. Though in general the user could specify any class/method in their program. 


\subsection{Inputs}

\jcupid chooses inputs automatically, attempting to find two inputs that produce
different execution traces. \jcupid can generate inputs in two ways. First, it
can generate a random string input, such as the ones used for our sample
programs. Second, \jcupid can work with jCute, a concolic execution testing tool
for Java programs that can statically determine a set of inputs to maximize code
coverage during execution. Unfortunately, we found that jCute was unable to
produce even obvious inputs for some of our toy programs, so we simply used our
former method, generating random strings of the same length as inputs to our
programs.


\subsection{Informing the Developer}

Once we find two inputs that produce different execution traces, \jcupid will
use the \texttt{-XX:+TraceBytecodes} flag, and find the differing bytecodes. It
then uses the \texttt{javap} utility in combination with the bytecode output to
determine what line of source code is responsible for producing those bytecodes.
Thus, we can inform the developer of the line in their code that is responsible
for this potential side channel, as well as the two inputs to their program that
cause it to diverge there.


%\begin{figure}[t]
%  \begin{center}
%    \begin{lstlisting}[language=make,caption={Example output from OpenJDK with trace flags set},label={lst:sample}]
%static void SumRandomBytes.main(jobject)
%     0  new 2 <java/util/Scanner>
%
%virtual jobject java.lang.ClassLoader.loadClass(jobject)
%     0  fast_aload_0
%     1  aload_1
%     2  iconst_0
%     3  invokevirtual 41 <java/lang/ClassLoader.loadClass(
%     Ljava/lang/String;Z)Ljava/lang/Class;> 
%    \end{lstlisting}
%  \end{center}
%\end{figure}

%\begin{figure}[h]
%  \begin{center}
%    \begin{lstlisting}[language=make,caption={Example output from \texttt{javap}},label={lst:lines}]
%public static void main(java.lang.String[]);
%    LineNumberTable:
%      line 31: 0
%      line 33: 11
%      line 35: 16
%      line 37: 21
%      line 40: 26
%      line 51: 32
%      line 42: 35
%      line 44: 37
%      line 45: 45
%      line 51: 49
%      line 47: 52
%      line 49: 54
%      line 50: 62
%      line 56: 66
%      line 57: 73
%      line 63: 81
%      line 59: 84
%      line 61: 86
%      line 62: 94
%      line 65: 98
%    \end{lstlisting}
%  \end{center}
%\end{figure}
%

%\subsection{OpenJDK Details}
%OpenJDK~\cite{OpenJDK} provided us with the ability to look at executed
%bytecodes. The key word here being \emph{executed} bytecodes, there are numerous
%static analyzers that allow the user to examine the bytecode information of
%their code~\cite{vallee1999soot}. However a bytecode (or timing) difference can
%occur in a library call and as such we need to dynamically analyze bytecodes
%that are executed on each run. The OpenJDK project is an open-source version of
%Oracle's JDK. This was key for \jcupid due to certain flags for the JDK
%(\texttt{-XX:+TraceBytecodes}) are only allowed in develop versions of the JDK,
%compiled with a debug flag. The release version of the JDK does not allow the
%use of this flag, however compiling OpenJDK with the debug flag allowed us
%access to dynamically executed bytecodes. Listing~\ref{lst:ex} shows an example
%of code in which bytecode differences will occur not in the source code but in a
%library call. The \texttt{SumRandomBytes} program will read a string from the
%user and then sum the bytes of the characters. This sum gives the number of
%bytes to read from a file which are then hashed. The hash function will do
%different amounts of work based on the given input string. 
%
%The above projects are great tools, but do not solve the problem at hand, which
%is to pinpoint which bytecode, or line of code causes a program do different
%work for different inputs. This is what \jcupid attempts to help solve with the
%aid, or incite of the above projects. 
%
%\begin{figure}[t]
%  \begin{center}
%    \begin{lstlisting}[caption={Example of code with bytecode difference in
%    Library call},label={lst:ex},language=Java] 
%public class SumRandomBytes
%{
%  public static void main(String [] args)
%  {
%    Scanner sc = new Scanner(System.in);
%    String s = sc.nextLine();
%
%    int sumOfBytes = sumString(s);
%    byte [] data = new byte[sumOfBytes];
%
%    data = readBytes(sumOfBytes);
%
%    MessageDigest md = MessageDigest.getInstance("SHA-1");
%    md.digest(data);
%  }
%}
%    \end{lstlisting}
%  \end{center}
%\end{figure}
