\documentclass{llncs}
\usepackage{epsfig,endnotes,url,listings}
\usepackage{color}

\definecolor{dkgreen}{rgb}{0,0.6,0}
\definecolor{gray}{rgb}{0.5,0.5,0.5}
\definecolor{mauve}{rgb}{0.58,0,0.82}

\lstset{frame=tb,
  language=Java,
  aboveskip=3mm,
  belowskip=3mm,
  showstringspaces=false,
  columns=flexible,
  basicstyle={\small\ttfamily},
  numbers=none,
  numberstyle=\tiny\color{gray},
  keywordstyle=\color{blue},
  commentstyle=\color{dkgreen},
  stringstyle=\color{mauve},
  breaklines=true,
  breakatwhitespace=true,
  tabsize=3
}

\PassOptionsToPackage{usenames,dvipsnames}{xcolor}
\usepackage{wrapfig}

\usepackage{amssymb}
\usepackage{amsfonts}
\usepackage{amsmath} 
\usepackage{tikz}
\usepackage{pgf}
\usepackage{pgflibraryarrows}
\usepackage{pgffor}
\usepackage{pgflibrarysnakes}

\newcommand{\jcupid}{\textsc{jCupid}}

\newcommand{\set}[1]{\left\{ #1 \right\}}
\newcommand{\Set}[1]{\big\{ #1 \big\}}
\newcommand{\seq}[1]{\langle #1 \rangle}
\newcommand{\Nat}{\mathbb N}
\newcommand{\Q}{\mathbb Q}
\newcommand{\R}{\mathbb R}
\newcommand{\Real}{\R}
\newcommand{\Int}{\mathbb Z}
\newcommand{\Rat}{\mathbb Q}
\newcommand{\Rplus}{\R_{\geq 0}}
\newcommand{\Zpos}{\Int_{\geq 0}}
\newcommand{\norm}[1]{\|#1\|}



\usepackage[colorlinks=true]{hyperref}
\usepackage{graphicx}
\usepackage{listings}

\begin{document}

\title{\textsc{jCupid}: A dynamic analysis tool for detecting side channels in Java programs}
\author{
  Ian Martiny
  \and
  Eric Wustrow
  \and
  Pavol {\v C}ern\'y
  \and
  Ashutosh Trivedi
}
\institute{University of Colorado Boulder, USA \\
  \email{\{ian.martiny, ewust, pavol.cerny, ashutosh.trivedi\}@colorado.edu}}


\maketitle

\begin{abstract}
  Side-channel based vulnerabilities continue to plague modern security-critical
  applications.
  From leaking private-keys to exposing sensitive data, 
  protecting against these threats remains a difficult task for developers. 
  A common defense against side-channels is to write constant-time code.
  However, ensuring that a complex program has no data-dependent control flow
  can be a laborious challenge, and can on occasion introduce or enable new side
  channel vulnerabilities.
  We present \jcupid, a tool that aids developers in detecting and
  removing a particular class of side channels from their code.
  \jcupid dynamically tests Java programs for timing side channels by examining
  the sequence of bytecodes that are executed for multiple inputs of the same size.
  Our tool optimizes this process by first computing only hashes of bytecode sequences,
  so that comparing the full sequences becomes necessary only when a side-channel has been detected.
  Once two inputs are found that cause a different sequence of bytecodes to be
  executed, \jcupid can inform the developer of the offending line of source code
  responsible for the divergence.
  While our tool can naively fuzz an instrumented program, it can also be used
  in conjuction with a Concolic execution testing engine.
  We show that instrumenting programs for use with \jcupid is simple for
  developers, and that our tool can reveal timing side-channel vulnerabilities
  that are not immediately obvious from source code inspection.
\end{abstract}



\section{Introduction}
\label{sec:introduction}
%% Timing side channels are the bane of developers for security programs. The
%% prototypical example of flawed program with a timing side channel is a password
%% checker which verifies user input against a stored password one character at a
%% time and immediately reports results. This program leaks timing information back
%% to the user -- on an incorrect password, how many of the characters were correct
%% (more correct characters means longer execution time). 

%% Often side channels are much less obvious, and much more subtle. Even solutions
%% in order to prevent side channels can create their own side
%% channels~\cite{al2013lucky}. However these (sometimes subtle) side channels can
%% leak valuable information, sometimes even private key information. 

%% Thus it must fall to the developer to ensure that no extra information is leaked
%% when using their software. Our work here is to aid the developer in this
%% difficult and precise task. In order to do so we re-framed the problem of timing
%% information to \emph{work done}. Timing side channels are what we aim to
%% eliminate with this tool, however we approach this in an oblique manner. Our
%% motivation for this is that timing is \emph{hard}, especially over networks or
%% systems with noise. This may add some false sense of security to developers that
%% this difficulty is a defense, however as networks and systems get less noisy
%% this ``defense'' diminishes. 

%% We must endeavor to remove these side channels, ideally before the code is even
%% used in a release. To get around the difficulty of timing imbalances we have
%% focused in on Java programs, and the work that they do. In particular Java
%% programs run instructions: bytecodes. Our tool, jCupid, will attempt to
%% determine if there are a set of inputs which cause a program to execute
%% different bytecodes. Essentially asking the question: are there inputs which
%% make programs do different work? 


%\subsection{Related Work}
%% Attacks
Side-channel attacks are the class of attacks on cryptographic system that
attempt to gain information about a system by by exploiting involuntary
information leaks via physical properties---
such as temperature, electromagnetic leaks, memory footprints, and timing
delays---of the implementation rather than using brute force or exploiting
theoretical weakness of the algorithm.
Timing-based side-channel attacks, a common class of
side-channel attacks, are extremely powerful as physical isolation of the
server does not help and it is easier to observe the timing differentials than
temperature, memory, or electromagnetic leaks.  
Some notable examples of timing-based information leaks include \emph{padding oracle
attack}~\cite{Vau02}, \emph{Lucky-13 attack}~\cite{al2013lucky},  \emph{modular
exponentiator} employed in Diffie-Hellman, RSA, and DSS~\cite{kocher96}.

\begin{algorithm}[t]
  \caption{A simple modular exponentiator algorithm with a secret-dependent side-channel vulnerability}
  \label{alg:expo}
  \begin{algorithmic}[1]
    \State  Let $R = 1$
    \For {$k = 0$ upto $w-1$}
    \State  $R = R^2 \mod n$
    \If {$x[k] = 1$}
    \State  $R := (R \cdot y) \mod n$
    \EndIf
    \EndFor
    \State \textbf{return} $R$
  \end{algorithmic}
\end{algorithm}

As an example of timing-based side-channel attack, consider modular
exponentiator used in the Diffie-Hellman and RSA cryptographic operations.
Here the system computes $R = y^x \mod n$ where $x$ is a secret key, $n$ is
public information, and for an attacker it is easy to get hold of $y$.
In this attack the value $x$ of the secret key is fixed, and the attacker can
observe system response time for various combinations of $y$ and $n$.
A simple modular exponentiator algorithm is shown as Algorithm~\ref{alg:expo}.
Observe that in this algorithm a timing side-channel exists: depending upon
the value of the secret input, line $5$ may or may not be executed, and hence
the algorithm will do different computation for different values of secret key.
This data-dependent side-channel has been shown by Kocher~\cite{kocher96} to be sufficient to recover the secret key.

A na\"ive remedy to timing-based information leaks in software-implemented
systems is to make the software run in fixed time for all possible inputs by
removing branches from code or
introducing timers to delay the observable outputs. 
However, this proves quite difficult to ensure in practice. For example, the Lucky 13
attack against the OpenSSL TLS implementation revealed the difficult nature of
removing subtle side-channels from secure code~\cite{al2013lucky}. At the time, OpenSSL's
ciphertext authentication was believed to be hardened against
side-channel attacks, due to its lack of branches that leaked information to a
remote attacker. However, there were subtle code paths that introduced branches
into the code unbeknownst to the developers. These branches came in the form of
how many bytes were hashed during verification of a decrypted payload, and were not easily
visible in the code on first inspection. Nonetheless, these minute differences
were enough to allow an attacker to perform a padding oracle attack and decrypt
arbitrary ciphertext. Patching the Lucky 13 attack took considerable work, in an
effort to remove all future potential timing attacks that might exist in this
code. This effort took multiple iterations, and included attempts that still
contained inadvertant and subtle side-channels~\cite{agl-lucky13}.

We present a tool, \jcupid, that helps developers discover
potential timing side-channel vulnerabilities in their Java code. \jcupid
functions by dynamically executing a provided program with same-length inputs,
and attempts to find two inputs that produce divergent bytecode executions at
runtime. By using bytecodes as a proxy for time, we eliminate the problem of
our timers being less accurate or fine-grained than that of potential
adversaries. Although our tool detects side-channels that may not be currently
practical to exploit (such as single bytecode differences in execution), time has shown
that attacks only get better. As these attacks improve, attackers are
able to use finer-resolution timers and techniques to discover subtle
differences in execution. Therefore, we are conservative and assume that even
tiny differences may be potentially observable.
\jcupid can be used to find simple common and subtle side-channels in several
programs, including modular exponentiation, password checking, and others. We
also test \jcupid against fixed versions of these programs, verifying that it
does not identify any ``false-positives''. With some developer effort, this tool
can help to reveal possible side-channels that may be present in a programmers
code.

\subsection{Related Work} 

Molnar et al.~\cite{Molnar05} proposed to model side-channels with
so-called \emph{transcript security model} where the sequence of certain
observable variable (transcripts) is leaked to the attacker.
Under this setting a program is secure if the adversary learns nothing about the
secret value even after given access to the side-channel transcripts.
Among others, program counters and opcodes have been proposed as an appropriate
notion of observable variables for transcripts.
Molnar et al.~\cite{Molnar05} present a run-time profiler for C programs to
detect side-channels by reporting pairs of inputs that differ in program-counter
transcripts.
In this paper we propose a similar runtime profiler to test whether a given
Java program is transcript secure where transcripts are Java bytecode
instructions for inputs of same size.
With some effort, our framework can be extended to more general transcripts.

Fuzz testing techniques~\cite{God12} are scalable and effective techniques for finding security
vulnerability in software.
Godefroid, Levin, and Molnar~\cite{God12} popularized the distinction between
blackbox and whitebox fuzz-testing techniques.
In blackbox fuzz-testing approach well-formed inputs are randomly modified,
while conforming to a template given as formal grammar and probabilistic
weights, to generate other potentially interesting inputs.  
Some popular tools for blackbox fuzzing include Peach 
(\url{http://www.peachfuzzer.com/}) and Autodafe
(\url{http://autodafe.sourceforge.net/}).
In contrast, whitebox fuzzing techniques---introduced by Godefroid et
al.~\cite{God12,GKS05}---exploit symbolic execution and dynamic test-case generation
to systematically generate test-cases to exercise different control-paths by
negating conditions exercised by previous test-cases.
CUTE and jCUTE~\cite{Sen2006} are popular whitebox fuzzing tools for C and Java
programs. 
The run-time profiling framework that we propose in this paper can be
effectively combined with both blackbox and whitebox fuzz testing techniques to
automate generation of interesting inputs.

Verification and static analysis techniques have been developed for
analysis of confidentiality properties and side channel information
leakage~\cite{EWS14,KMO12,BSB07,BDG12,CA09}. The strength of these
works lies in the fact that they provide a specification language that
allows the developer to specify the confidentiality requirement
tailored to the application. In contrast, our tool has one
(generic) notion of confidentiality, but is much more lightweight and
scalable. 

An issue related to testing and verification of transcript security of
a program is the automatic synthesis of transcript-secure programs
from un-secure ones. Eldib and Wang~\cite{EW14} proposed a synthesis
method (and tool SC Masker) based on LLVM compiler and Yices SMT
solver to produce a perfectly-masked 
program such that resulting program is such that all the observable outputs are
statistically independent from the input data. 
Prouff and Rivain~\cite{PR13} outline a formal security proof for masked
implementations of block ciphers.      

For a background on formal quantitative information flow we refer the reader to
an excellent introductory article by Smith~\cite{smith09} and references therein.
Static and dynamic information flow techniques~\cite{SR10,Den76} have concentrated
mainly upon explicit and implicit information flow from secret variables to
outputs or observable variables.
However, side-channel information is often not explicitly present in the text
(source code or bytecode) of the program.
A key challenge under this assumption is to make this timing side-channel
explicit by learning timing function of various program units in the software
and use them to either statically analyze the software for side-channel freedom,
or use run-time profilers to execute inputs of similar size such that the
observable input differs.
The tool \jcupid reported in this paper is an attempt in this direction.

%% \subsection{Old Part}
%% The idea monitoring executed bytecodes is not dissimilar to that of finding all
%% of the paths through a program. This led us to many concolic testers and input
%% fuzzers.  

%% In particular jFuzz~\cite{jayaraman2009jfuzz}, a concolic whitebox input fuzzer,
%% built on the NASA Java PathFinder. The goal of jFuzz is to start from a given
%% seed and using this find inputs which lead down all possible paths of a given
%% program. As mentioned this goal is not completely perpendicular to our own, and
%% finding which inputs lead down unique paths is valuable information. This
%% project however was difficult to get running, as it requires Java 1.5, and
%% seemingly does not work with Java 1.8. 

%% Additionally another concolic tester, jCute~\cite{conf/cav/SenA06} allowed us to
%% use more recent versions of Java. While jCute was a promising tool we found that
%% that for some of our test programs jCute did not recognize various paths in our
%% program. 





\section{\textsc{jCupid}}
\label{sec:jcupid}
jCupid is at heart a python script which takes advantage of OpenJDK. Besides allowing us to see the bytecodes as they are executed we were able to modify OpenJDK in various ways to help get results quickly. When running jCupid we first need to determine if there are different inputs which run different bytecodes. However a lot of bytecodes are run even for very simple programs, for a ``Hello World'' program, running OpenJDK with the \texttt{-XX:+TraceBytecodes} flag takes 8.6 seconds with 1,179,829 bytecodes executed. Even more disheartening is that even this simple program does not run the same bytecodes in the same order for consecutive runs. Most of the bytecodes that are executed are just for the JVM start up and shutdown. In fact only 6,054 bytecodes are run from when the \texttt{main} method begins to when it returns. Meaning that 1,173,775 bytecodes that are executed are outside of the user's control. In order to avoid examining all of these bytecodes and finding numerous false positive bytecode differences we modified OpenJDK to allow for us to dictate which bytecodes we will examine. We will discuss this more in Section~\ref{sec:OpenJDK}.

After determining two inputs that lead to different bytecodes being executed we need to trace this back to a source code line number -- it is no help to just inform developers there is a problem they need to know what to fix. Luckily the flag \texttt{-XX:+TraceBytecodes} and the \texttt{javap} utility provide a way to do this. We now dive into the execution of the jCupid tool.

\subsection{Inputs}

First we need to discuss the creation of inputs to sample programs. The goal of this project is to find different bytecodes being executed as a result of different inputs, so it is natural to want to find all different paths in a program. As mentioned above jCute would be a great resource for this. However there were sample programs where jCute was unable to determine paths which were dependent about input. 

Another flaw is just finding paths may not be what we are looking for. For example even the best implementation of AES is going to take a different amount of time and run different bytecodes when trying to encrypt a message of one byte vs. a message of a million bytes. However the interesting question is are there any messages of length 128 bytes, say, which cause a particular implementation of AES to run different bytecodes. 

Thus finding inputs that lead down different paths is interesting however we would need to enforce that the inputs we also of a particular form. For jCupid we decided to have the user provide the input size and jCupid would generate random strings in an attempt to get different bytecodes to execute.

\subsection{OpenJDK}\label{sec:OpenJDK}

Modifying OpenJDK was a major part of this project. As mentioned above it was slow to run the desired flags over a program and wait for the output. Since we are trying random inputs we do not want to have to wait for every input to list their bytecodes and then compare them, as well as ignore any false positives.

To help in this we modified OpenJDK to allow for new flags: \texttt{-hashClass}, \texttt{-hashMethod}, \texttt{-traceClass}, and \texttt{-traceMethod}. Each takes an argument, a class name or method name respectively. The \texttt{-hashClass} and \texttt{-hashMethod} flags are used together as are the trace variants. Their arguments dictate the class and method that we are interested in. For example in Listing~\ref{lst:ex} we would be interested in the class \texttt{SumRandomBytes} and the method \texttt{main}. Though in general the user could specify any class/method in their program. 

When the hash flags are used, instead of printing all bytecodes that are executed the modified JVM will execute bytecodes until it reaches the method specified and then begin iteratively hashing bytecodes, using Berstein's djb2 hash~\cite{djb2Hash}, until the user selected method's  return statement is executed, and then simply execute the remaining bytecodes (again without printing). Finally the JVM will print the resulting hash. 

These hash flags provide a very quick way to verify whether the same bytecodes have been executed or not. The ``Hello World'' program mention above which took 8.6 seconds to run when printing the bytecodes takes only 0.78 seconds to run with the hash flags. Additionally the hash result is a simple check to see if same bytecodes are being executed.

Once a new hash is seen we have two inputs which lead to different bytecodes being executed. Now the goal is to determine which bytecodes differ and which line of source code this corresponds to. This is helped by the trace flags above. The JVM is run with the trace flags and again executes bytecodes until the specified method is run then it begin printing bytecodes and stops once the specified method returns. This is essentially what \texttt{-XX:+TraceBytecodes} does but is much faster (the ``Hello World'' program runs in about 0.78 seconds again) and has the benefit that it ignores the start up and shut down of the JVM which will eliminate false positives.

\subsection{Cross-referencing to line number}

Once we have run the JVM with the trace flags we can compare the outputs to determine where the first difference occurs. However this will not directly give us a line number in the source code. As mentioned above the output from the trace flags as well as the \texttt{javap} utility will be of help here. A sample (filtered) output from \texttt{-XX:+TraceBytecodes} is included in Listing~\ref{lst:sample}, the numbers that are listed before bytecodes is the relative order the bytecode is executed within the listed function. For one this is helpful while reading this file to determine where in the function we are executing. Since this is the dynamic execution order of bytecodes the output often jumps between functions at various places. Much more helpful is the fact that this relative bytecode count is also listed by line number with the \texttt{javap} utility, as seen in Listing~\ref{lst:lines}. Thus the last stage of jCupid is to look at this material and look for the last line of the source code where the bytecodes agree and inform the user that the inputs do different work after this point.

\begin{figure}[t]
  \begin{center}
    \begin{lstlisting}[language=make,caption={Example output from OpenJDK with trace flags set},label={lst:sample}]
static void SumRandomBytes.main(jobject)
     0  new 2 <java/util/Scanner>

virtual jobject java.lang.ClassLoader.loadClass(jobject)
     0  fast_aload_0
     1  aload_1
     2  iconst_0
     3  invokevirtual 41 <java/lang/ClassLoader.loadClass(
     Ljava/lang/String;Z)Ljava/lang/Class;> 
    \end{lstlisting}
  \end{center}
\end{figure}

\begin{figure}[h]
  \begin{center}
    \begin{lstlisting}[language=make,caption={Example output from \texttt{javap}},label={lst:lines}]
public static void main(java.lang.String[]);
    LineNumberTable:
      line 31: 0
      line 33: 11
      line 35: 16
      line 37: 21
      line 40: 26
      line 51: 32
      line 42: 35
      line 44: 37
      line 45: 45
      line 51: 49
      line 47: 52
      line 49: 54
      line 50: 62
      line 56: 66
      line 57: 73
      line 63: 81
      line 59: 84
      line 61: 86
      line 62: 94
      line 65: 98
    \end{lstlisting}
  \end{center}
\end{figure}


\subsection{OpenJDK Details}
OpenJDK~\cite{OpenJDK} provided us with the ability to look at executed
bytecodes. The key word here being \emph{executed} bytecodes, there are numerous
static analyzers that allow the user to examine the bytecode information of
their code~\cite{vallee1999soot}. However a bytecode (or timing) difference can
occur in a library call and as such we need to dynamically analyze bytecodes
that are executed on each run. The OpenJDK project is an open-source version of
Oracle's JDK. This was key for jCupid due to certain flags for the JDK
(\texttt{-XX:+TraceBytecodes}) are only allowed in develop versions of the JDK,
compiled with a debug flag. The release version of the JDK does not allow the
use of this flag, however compiling OpenJDK with the debug flag allowed us
access to dynamically executed bytecodes. Listing~\ref{lst:ex} shows an example
of code in which bytecode differences will occur not in the source code but in a
library call. The \texttt{SumRandomBytes} program will read a string from the
user and then sum the bytes of the characters. This sum gives the number of
bytes to read from a file which are then hashed. The hash function will do
different amounts of work based on the given input string. 

The above projects are great tools, but do not solve the problem at hand, which
is to pinpoint which bytecode, or line of code causes a program do different
work for different inputs. This is what jCupid attempts to help solve with the
aid, or incite of the above projects. 

\begin{figure}[t]
  \begin{center}
    \begin{lstlisting}[caption={Example of code with bytecode difference in
    Library call},label={lst:ex},language=Java] 
public class SumRandomBytes
{
  public static void main(String [] args)
  {
    Scanner sc = new Scanner(System.in);
    String s = sc.nextLine();

    int sumOfBytes = sumString(s);
    byte [] data = new byte[sumOfBytes];

    data = readBytes(sumOfBytes);

    MessageDigest md = MessageDigest.getInstance("SHA-1");
    md.digest(data);
  }
}
    \end{lstlisting}
  \end{center}
\end{figure}


\section{Evaluation}
\label{sec:evaluation}


We evaluated jCupid using a set of simple programs that commonly contain timing
side channels in their implementations.


\paragraph{Password checking}
The na\"{\i}ve method of checking whether a provided password is correct is to iterate
over the provided input, comparing each character to its expected value, and
returning failure on the first discovered difference. This results in a side
channel that leaks the number of characters that an attacker has guessed
correctly, allowing them to perform an adaptive attack. The attacker can guess
each character independently, and only move on to the next character when they
have found the correct character in the current position. While we call this a
password checking test case, this pattern also occurs in any string comparison
that should be done in constant time, such as when checking if CSRF or
authentication tokens are valid.

%\paragraph{Lucky-13 attack}
%We implemented a toy Java version of the padding and MAC check employed in TLS
%decryption. Our implementation was vulnerable to the Lucky 13 attack, where
%an attacker modifies ciphertext, and can distinguish between the corresponding
%plaintext having valid padding or not. This capability allows an attacker to
%perform a Vaudenay attack and iteratively leak the plaintext.

\paragraph{Modular Exponentiation}
We implemented the straightforward method for modular exponentiation shown in
Algorithm~\ref{alg:expo}. This algorithm has a side-channel dependent on the
(private) exponent input. To compare against a side-channel-free version,
we also implemented a variant on the Montgomery Ladder
technique for modular exponentiation. However, we had to use algebric tricks to
replace the branches that are present in the algorithm. Our updated version has no
data-dependent branches, and relies only on multiplication, addition,
subtraction, and modulo.


\paragraph{SumBytes}
We created a program that reads a single byte to determine a length, then reads
that many bytes and hashes them. Here, we are mimicking a simple serialized
network protocol with a length encoded at the beginning of the message. The side
channel present in this code is subtle: there are no if statements or obvious
branches in the code. Rather, the number of bytes read determines the number of
compression function iterations occur internal to our hash function.

\paragraph{Multiplexer}
Finally, we implemented a selection function that takes three inputs: a, b, and
a selector. If the selector is true, our function returns the first input,
otherwise it returns the second input. This is easily accomplished by an if
statement, however, such an implementation may leak information due to instruction cache timing or
memory accesses.



For each of these programs, we allowed them to take a single input as a string
on standard input. Then, we allowed \jcupid to fuzz these programs with random
inputs, and look for potential differences in the bytecodes executed. In each
example, we found the intended side-channel, and \jcupid was able to correctly
determine the line of code responsible.

We then corrected the problems identified by \jcupid, and reran our tool to
verify we had removed the offending side-channels. To our surprise, \jcupid
identified additional problems in some of our ``corrected'' programs. For example, in
the password checking program, we used a temporary value, and updated it at each
iteration with $ good \&= (input[i] == expected[i]) $. However, this resulted in
a side-channel when computing the $ == $ operator, as Java implemented this as a
branch in the bytecode. We fixed this by switching to using exclusive-or to
compare the values, and \jcupid did not detect any additional side-channels. This
further illustrates the difficulty of removing all potential side-channels from
code: even seemingly branch-free programs, written with the intention of not
having side-channels, can contain them.


As \jcupid is a dynamic analysis tool, it must run many instances of the program
in order to find side-channel behavior. We evaluated \jcupid's overhead by
measuring how quickly it can run basic Java programs, and compared it to
running those same programs outside of the \jcupid environment. Particularly for
a developer dynamic analysis tool that inspects individual bytecodes, our
results are encouraging: per run, \jcupid only adds on average approximatly 2x
overhead. This could allow even rare side-channels to be detected with
na\"{\i}ve fuzzing of nightly builds over unit tests.

\paragraph{Timing experimentation}
In order to test the efficiency of \jcupid, we ran a number of experiments to
determine how much overhead \jcupid has, and where the overhead comes from.
With the above test programs, we ran each program 10 times with an unmodified
OpenJDK. Then, we ran each program using our tool with 10 iterations. Over all
of the runs, we found that stock OpenJDK took on average 2.6 seconds per
iteration, while \jcupid took on average 5.3 seconds per iteration. However, this
is obviously variable based on the input program: for longer running and more
computationally-expensive programs, \jcupid's overhead is lower.

To determine which components of \jcupid are responsible for its overhead, we
removed individual components and reran our programs. We ran our python tool in
a way that it called a normal OpenJDK rather than our modified (debug) one, and
found that it took on average 3.6 seconds per run. We then ran our tool using a
modified version of OpenJDK that did not use hashing, and found it takes 4.8
seconds. Thus, we estimate that our python tool adds about half of the overhead
of \jcupid,
with the remainder coming mainly from the debug version of OpenJDK, and a small
contribution from the fast hash function itself.

%with the inputs. Then we ran the programs with a modified version of our tool 
%which simply calls OpenJDK, with no flags -- this allows us to test how much 
%overhead the tool itself has. Finally we run the tool on a modified OpenJDK such
%that the hash function does no work (meaning nothing will be flagged) -- this 
%tells us how much time our hash function is taking. Each of the above tests are 
%run 5 times.
%
%Averaging the time results across the different programs provides the following
%results: just calling OpenJDK takes around 2.6 seconds, \jcupid averages 5.3 seconds
%per call to OpenJDK. This is an overhead of 2.7 seconds. The following two results
%will help us determine where this overhead comes from. With the modified jCupid tool
%that just calls OpenJDK with no flags takes 3.6 seconds, and finally calling jCupid
%that calls a modified OpenJDK that does not do hashing takes 4.8 seconds.
%
%The above results tell us that of the 2.7 seconds of overhead 1.0 seconds of overhead
%come from the tool itself, and 1.2 second of overhead comes from the flags that
%we use in jCupid. Thus the remaining 0.5 seconds of overhead come from the hash
%function itself.


\section{Future Work}
\label{sec:future}

There are many avenues of this project to advance work. In particular finding, or adapting, a concolic tester to hit all paths in a program. For the examples that jCute did work on, it was much faster at finding input than the random fuzzer that we implemented (as would be expected). As mentioned above not only would we want this tester to be able to find all paths, but find paths that are reachable under given constraints, such as input length.

Another interesting avenue is to refine this tool to examine the types of differences we see in bytecode execution. In particular it may be that certain programs are written so that there are not timing difference but there are bytecode execution differences. In the goal of this project to correct timing side channels this might be considered a false positive as well. Thus it would be interesting to collect bytecodes into equivalence classes by time and allowing differences in execution within these classes.

This project was written for Java programs, it would be a note-worthy extension to continue this project in C to let a greater audience access to this tool.


\section{Conclusion}
\label{sec:conclusion}
Timing side-channels are a great concern when developing 
secure programs. With \jcupid, we hope to aid developers in writing
code that is timing side-channel free. In order to find the side
channels, \jcupid 
tracks the sequence of bytecodes that are executed for multiple inputs
of the same size. Once two inputs are found that cause a different
sequence of bytecodes to be executed, \jcupid can inform the developer
of the line that caused the
divergence. We show that in this way \jcupid can detect side channel
bugs using a series of case studies (and we show that \jcupid does not
detect bugs in the corrected versions).

This tool is intended to be used by developers to test very
specific aspects of their code. Running \jcupid on even moderately sized
projects, say an authentication server, will most
certainly find different bytecodes executed based on input (even of
the same length). \jcupid is intended as a more precise tool. Instead of 
analyzing the whole server, the tool is
intended to analyze whether the modular exponentiation function
executes different bytecodes for inputs of the same size (and thus for
instance reveals information about bits in the secret key). 



\bibliographystyle{plain}
\bibliography{papers}

\newpage
\appendix
% Appendix 

Presenter (P): jCupid is a python script wrapped around a modified version of OpenJDK. jCupid's goal
is to assist developers in fixing timing vulnerabilities in their code. The user would provide a sample
Java program and specific method to test for timing vulnerabilities. jCupid would then choose various
inputs to give to the program and examine the bytecodes that are executed.

As opposed to timing the the run of the program jCupid actually analyses the bytecodes that the program
executes, the idea being that a given program should do exactly the same instructions for all input.

jCupid achieves this by using some functionality of the debug version of the OpenJDK as well as some
modifications that were made. Namely jCupid will provide OpenJDK with the name of the class and method
the user is interested in, then from the time that the method begins execution until it returns OpenJDK
will run a quick hash of all of the bytecodes that are executed. This provides a quick way of determining
whether the inputs are doing different work, if they produce different hashes, then jCupid will dive 
in more in depth into those executions, but inputs that lead to the same hash do not need extra time
spent on them. As mentioned after different hashes are produced by inputs those inputs are provided
to the program again with different flags for OpenJDK that will provide a list of all the bytecodes 
executed (in order). Then jCupid will then compare those outputs to find the bytecode that is different
and finally work to find the line of source code that presented this difference.

When calling jCupid the user must provide certain information namely: the filename of the java program,
the name of the class and method that the user is interested in testing, the length of inputs to provide
to the program and the maximum number of iterations to run. Thus a potential call to jCupid would look
as such:

\begin{center}
  \includegraphics[width=\linewidth]{jCupidCall}
\end{center}

Which tells jCupid that it should use at most 6 random strings of length 5 to run the file PasswordChecker.java
and to examine the method checkPass in the class PasswordChecker. A quick look at that function tells us how
it will check passwords:

\begin{center}
  \includegraphics[width=\linewidth]{PasswordChecker}
\end{center}

As we can see this is the simple way of checking passwords, one character at a time and immediately reporting
any failures. This definitely would lead to a timing vulnerability. For reference the password is 
``mySecretPassword'', so we should see some different bytecodes executed between inputs that start with `m'
vs ones that don't. Lets see how jCupid handles this:

\begin{center}
  \includegraphics[width=\linewidth]{jCupidRun1}
\end{center}

Here we see that jCupid tells us what it is inputting. The first input is ``VNzmP'', the first character
isn't `m' so we know it will fail immediately. jCupid then shows the output from the program and some
diagnostic output including how many bytecodes were executed and the result of the hash. The second input
is ``8A0cG'' which again should fail immediately and we can see it has the same hash and moves on to the
next input. The third input is ``mEL1G'', this input should lead us down a different path since the first
character matches the first character of our password. Indeed we can see that the hash for this input is
different. This prompts jCupid to now look for the offending bytecode and trace this back to a line in 
the source code, in this case it tells us line 21 is the offending line, which we can see is the line of
the \texttt{if} statement.

The now informed developer can recognize their mistake and fix this code! One common way of ``correctly''
checking passwords is to always loop through the provided password and simply update a counter by and-ing
whether each character matches the correct password. This implementation can be seen here:

\begin{center}
  \includegraphics[width=\linewidth]{PasswordCheckerFix1}
\end{center}

This fix looks good! We always examine the full length of the input and are simply updating a variable
with the exact same line of code no matter what. Lets see what jCupid says:

\begin{center}
  \includegraphics[width=\linewidth]{jCupidRun2}
\end{center}

Running the same inputs through again we see that indeed the two first inputs do the same work, as expected,
but suprisingly we still get different bytecodes executed with this fixed solution. After much investigation
it appears that a peculiarity of Java turns the $==$ into an if statement! Luckily we ran the code through
jCupid again. After some clever thinking we may realize that we need to do this checking without actually
comparing the characters, but we can do a computation on them. Specifically we can xor each character to
determine if they are the same, and then keep a running \texttt{or} to keep track of any differences showing
up. So the code now looks like:

\begin{center}
  \includegraphics[width=\linewidth]{PasswordCheckerFix2}
\end{center}

Now there are no decisions being made at all in the code, just a computation being done, and the same 
computation. Lets see if jCupid agrees with us:

\begin{center}
  \includegraphics[width=\linewidth]{jCupidRun3}
\end{center}

Indeed we can see that jCupid found no difference between any of the inputs, regardless of whether they 
began the same as the password. Of course we would want to run significantly more trials on numerous 
length inputs in order to be more confident, but we can see that we have fixed our problem of matching
letters causing different instructions to be executed. Even though the first solution we had seemed like
it was exactly what we wanted, Java's compiler changes the code in subtle ways that can add side-channels.

It is worth noting that jCupid can be fooled by using some built-in java libraries. Examine this version
of checking passwords, by simply comparing strings:

\begin{center}
  \includegraphics[width=\linewidth]{PasswordCheckerFix3}
\end{center}

Why bother going through the process of comparing strings yourself when Java provides a built-in call for
this? Lets see what jCupid thinks:

\begin{center}
  \includegraphics[width=\linewidth]{jCupidRun4}
\end{center}

It appears that all is well! jCupid doesn't notice any difference by using Java's built-in string comparison!
However after some investigation: a significant portion of Java's built-in calls are implemented in \texttt{C++}
meaning that jCupid can't detect any differences, since it only looks at the bytecodes that Java executes. And
in fact Java's built-in methods are not designed to be used in secure applications.

We provide another example: sometimes users code may not have any branches at all, but still input has an effect
on what is computed. In certain situations some inputs can cause more work to be done. In particular consider
the contrived program that reads a string from input and then sums the characters of the string to determine
how many bytes to read from a file and then hash.

Even two inputs of the same size can lead to different bytecodes being executed. So our sample program has
the following structure:

\begin{center}
  \begin{lstlisting}[language=Java]
  public class SumRandomBytes
  {
    public static void main(String [] args)
     {
       Scanner sc = new Scanner(System.in);
       String s = sc.nextLine();
      
       int sumOfBytes = sumString(s);
       byte[] data = new byte[sumOfBytes];
    
       data = readBytes(sumOfBytes);
 
       MessageDigest md = MessageDigest.getInstance(SHA-1);
       md.digest(data);
    }
 }
 \end{lstlisting}
\end{center}

jCupid catches this as:

\begin{center}
  \includegraphics[width=\linewidth]{jCupidSumRandomBytes}
\end{center}

Line 34 corresponds to the \texttt{md.digest(data)} line, which tells us that something is wrong with our
hashing. Of course the issue come from the inputs: \texttt{44LUF} sums to 335 and \texttt{BvHBJ} sums to 
396, thus the first run of our program hashes 335 bytes of data where as the second run hashes 396 bytes,
which leads to a different number of bytecodes executed.


\end{document}

