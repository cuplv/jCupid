%% Attacks
Side-channel attacks are the class of attacks on cryptographic system that
attempt to gain information about a system by by exploiting involuntary
information leakage via physical properties---
such as temperature, electromagnetic leaks, memory footprint, and timing
delays---of the implementation rather than using brute force or exploiting
theoretical weakness of the algorithm.
%Perhaps the earliest documented example of a side-channel attack 
%exploit 
A number of known cryptographic attacks such as Padding Oracle
attacks~\cite{Vau02}, Lucky-13 attack~\cite{al2013lucky}, 

%% Testers -- concolic -- fuzzers
Concolic testing~\cite{GKS05} and CUTE and jCUTE~\cite{Sen2006}.


%% Side-Channnel Detection --Wao Chang Kopf
Synthesis of masking countermeasures against side-channel attacks~\cite{EW14}


%% Info Flow
secure information flow~\cite{Den76}, TaintDroid~\cite{Enck14}, Dynamic and
Static Information Flow~\cite{SR10}.

The idea monitoring executed bytecodes is not dissimilar to that of finding all
of the paths through a program. This led us to many concolic testers and input
fuzzers.  

In particular jFuzz~\cite{jayaraman2009jfuzz}, a concolic whitebox input fuzzer,
built on the NASA Java PathFinder. The goal of jFuzz is to start from a given
seed and using this find inputs which lead down all possible paths of a given
program. As mentioned this goal is not completely perpendicular to our own, and
finding which inputs lead down unique paths is valuable information. This
project however was difficult to get running, as it requires Java 1.5, and
seemingly does not work with Java 1.8. 

Additionally another concolic tester, jCute~\cite{conf/cav/SenA06} allowed us to
use more recent versions of Java. While jCute was a promising tool we found that
that for some of our test programs jCute did not recognize various paths in our
program. 

Also OpenJDK~\cite{OpenJDK} provided us with the ability to look at executed
bytecodes. The key word here being \emph{executed} bytecodes, there are numerous
static analyzers that allow the user to examine the bytecode information of
their code~\cite{vallee1999soot}. However a bytecode (or timing) difference can
occur in a library call and as such we need to dynamically analyze bytecodes
that are executed on each run. The OpenJDK project is an open-source version of
Oracle's JDK. This was key for jCupid due to certain flags for the JDK
(\texttt{-XX:+TraceBytecodes}) are only allowed in develop versions of the JDK,
compiled with a debug flag. The release version of the JDK does not allow the
use of this flag, however compiling OpenJDK with the debug flag allowed us
access to dynamically executed bytecodes. Listing~\ref{lst:ex} shows an example
of code in which bytecode differences will occur not in the source code but in a
library call. The \texttt{SumRandomBytes} program will read a string from the
user and then sum the bytes of the characters. This sum gives the number of
bytes to read from a file which are then hashed. The hash function will do
different amounts of work based on the given input string. 

The above projects are great tools, but do not solve the problem at hand, which
is to pinpoint which bytecode, or line of code causes a program do different
work for different inputs. This is what jCupid attempts to help solve with the
aid, or incite of the above projects. 

\begin{figure}[t]
  \begin{center}
    \begin{lstlisting}[caption={Example of code with bytecode difference in
    library call},label={lst:ex},language=Java] 
public class SumRandomBytes
{
  public static void main(String [] args)
  {
    Scanner sc = new Scanner(System.in);
    String s = sc.nextLine();

    int sumOfBytes = sumString(s);
    byte [] data = new byte[sumOfBytes];

    data = readBytes(sumOfBytes);

    MessageDigest md = MessageDigest.getInstance("SHA-1");
    md.digest(data);
  }
}
    \end{lstlisting}
  \end{center}
\end{figure}
