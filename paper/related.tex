%% Attacks
Side-channel attacks are the class of attacks on cryptographic system that
attempt to gain information about a system by by exploiting involuntary
information leaks via physical properties---
such as temperature, electromagnetic leaks, memory footprints, and timing
delays---of the implementation rather than using brute force or exploiting
theoretical weakness of the algorithm.
Timing-based side-channel attacks, a common class of
side-channel attacks, are extremely powerful as physical isolation of the
server does not help and it is easier to observe the timing differentials than
temperature, memory, or electromagnetic leaks.  
Some notable examples of timing-based information leaks include \emph{padding oracle
attack}~\cite{Vau02}, \emph{Lucky-13 attack}~\cite{al2013lucky},  \emph{modular
exponentiator} employed in Diffie-Hellman, RSA, and DSS~\cite{kocher96}.

As an example of timing-based side-channel attack, consider modular
exponentiator used in the Diffie-Hellman and RSA cryptographic operations.
Here the system computes $R = y^x \mod n$ where $x$ is a secret key, $n$ is
public information, and for an attacker it is easy to get hold of $y$.
In this attack the value $x$ of the secret key is fixed, and the attacker can
observe system response time for various combinations of $y$ and $n$.
A simple modular exponentiator algorithm is shown as Algorithm~\ref{alg:expo}.
Observe that in this algorithm a timing side-channel exists: depending upon
the value of the secret input, line $5$ may or may not be executed, and hence
the algorithm will do different computation for different values of secret key.
This data-dependent side-channel has been shown by Kocher~\cite{kocher96} to be sufficient to recover the secret key.
\begin{algorithm}[t]
Let $R = 1$\;
\For {$k = 0$ upto $w-1$}{
$R = R^2 \mod n$\;
\If {$x[k] = 1$}{
$R := (R \cdot y) \mod n$\;
}
}
\Return $R$\;
\caption{A simple modular exponentiator algorithm with a secret-dependent side-channel vulnerability}
\label{alg:expo}
\end{algorithm}

A na\"ive remedy to timing-based information leaks in software-implemented
systems is to make the software run in fixed time for all possible inputs by
removing branches from code or
introducing timers to delay the observable outputs. 
However, this proves quite difficult to ensure in practice. For example, the Lucky 13
attack against the OpenSSL TLS implementation revealed the difficult nature of
removing subtle side-channels from secure code~\cite{al2013lucky}. At the time, OpenSSL's
ciphertext authentication was believed to be hardened against
side-channel attacks, due to its lack of branches that leaked information to a
remote attacker. However, there were subtle code paths that introduced branches
into the code unbeknownst to the developers. These branches came in the form of
how many bytes were hashed during verification of a decrypted payload, and were not easily
visible in the code on first inspection. Nonetheless, these minute differences
were enough to allow an attacker to perform a padding oracle attack and decrypt
arbitrary ciphertext. Patching the Lucky 13 attack took considerable work, in an
effort to remove all future potential timing attacks that might exist in this
code. This effort took multiple iterations, and included attempts that still
contained inadvertant and subtle side-channels~\cite{agl-lucky13}.

In this paper we present a tool, \jcupid, that helps developers discover
potential timing side-channel vulnerabilities in their Java code. \jcupid
functions by dynamically executing a provided program with same-length inputs,
and attempts to find two inputs that produce divergent bytecode executions at
runtime. By using bytecodes as a proxy for time, we eliminate the problem of
our timers being less accurate or fine-grained than that of potential
adversaries. Although our tool detects side-channels that may not be currently
practical to exploit (such as single bytecode differences in execution), time has shown
that attacks only get better. As these attacks improve, attackers are
able to use finer-resolution timers and techniques to discover subtle
differences in execution. Therefore, we are conservative and assume that even
tiny differences may be potentially observable.

We use \jcupid to find simple common and subtle side-channels in several
programs, including modular exponentiation, password checking, and others. We
also test \jcupid against fixed versions of these programs, verifying that it
does not identify any ``false-positives''. With some developer effort, this tool
can help to reveal possible side-channels that may be present in a programmers
code.

\subsection{Related Work} 

Molnar et al.~\cite{Molnar05} proposed to model side-channels with
so-called \emph{transcript security model} where the sequence of certain
observable variable (transcripts) is leaked to the attacker.
Under this setting a program is secure if the adversary learns nothing about the
secret value even after given access to the side-channel transcripts.
Among others, program counters and opcodes have been proposed as an appropriate
notion of observable variables for transcripts.
Molnar et al.~\cite{Molnar05} present a run-time profiler for C programs to
detect side-channels by reporting pairs of inputs that differ in program-counter
transcripts.
In this paper we propose a similar runtime profiler to test whether a given
Java program is transcript secure where transcripts are Java bytecode
instructions for inputs of same size.
With some effort, our framework can be extended to more general notions of
transcripts.

Fuzz testing techniques~\cite{God12} are scalable and effective techniques for finding security
vulnerability in software.
Godefroid, Levin, and Molnar~\cite{God12} popularized the distinction between
blackbox and whitebox fuzz-testing techniques.
In blackbox fuzz-testing approach well-formed inputs are randomly modified,
while conforming to a template given as formal grammar and probabilistic
weights, to generate other potentially interesting inputs.  
Some popular tools for blackbox fuzzing include Peach 
(\url{http://www.peachfuzzer.com/}) and Autodafe
(\url{http://autodafe.sourceforge.net/}).
In contrast, whitebox fuzzing techniques---introduced by Godefroid et
al.~\cite{God12,GKS05}---exploit symbolic execution and dynamic test-case generation
to systematically generate test-cases to exercise different control-paths by
negating conditions exercised by previous test-cases.
CUTE and jCUTE~\cite{Sen2006} are popular whitebox fuzzing tools for C and Java
programs. 
The run-time profiling framework that we propose in this paper can be
effectively combined with both blackbox and whitebox fuzz testing techniques to
automate generation of interesting inputs.

Verification and static analysis techniques have been developed for
analysis of confidentiality properties and side channel information
leakage~\cite{EWS14,KMO12,BSB07,BDG12,CA09}. The strength of these
works lies in the fact that they provide a specification language that
allows the developer to specify the confidentiality requirement
tailored to the application. In contrast, our tool has one
(generic) notion of confidentiality, but is much more lightweight and
scalable. 

An issue related to testing and verification of transcript security of
a program is the automatic synthesis of transcript-secure programs
from un-secure ones. Eldib and Wang~\cite{EW14} proposed a synthesis
method (and tool SC Masker) based on LLVM compiler and Yices SMT
solver to produce a perfectly-masked 
program such that resulting program is such that all the observable outputs are
statistically independent from the input data. 
Prouff and Rivain~\cite{PR13} outline a formal security proof for masked
implementations of block ciphers.      

For a background on formal quantitative information flow we refer the reader to
an excellent introductory article by Smith~\cite{smith09} and references therein.
Static and dynamic information flow techniques~\cite{SR10,Den76} have concentrated
mainly upon explicit and implicit information flow from secret variables to
outputs or observable variables.
However, side-channel information is often not explicitly present in the text
(source code or bytecode) of the program.
A key challenge under this assumption is to make this timing side-channel
explicit by learning timing function of various program units in the software
and use them to either statically analyze the software for side-channel freedom,
or use run-time profilers to execute inputs of similar size such that the
observable input differs.
Our tool is an attempt in this direction.

%% \subsection{Old Part}
%% The idea monitoring executed bytecodes is not dissimilar to that of finding all
%% of the paths through a program. This led us to many concolic testers and input
%% fuzzers.  

%% In particular jFuzz~\cite{jayaraman2009jfuzz}, a concolic whitebox input fuzzer,
%% built on the NASA Java PathFinder. The goal of jFuzz is to start from a given
%% seed and using this find inputs which lead down all possible paths of a given
%% program. As mentioned this goal is not completely perpendicular to our own, and
%% finding which inputs lead down unique paths is valuable information. This
%% project however was difficult to get running, as it requires Java 1.5, and
%% seemingly does not work with Java 1.8. 

%% Additionally another concolic tester, jCute~\cite{conf/cav/SenA06} allowed us to
%% use more recent versions of Java. While jCute was a promising tool we found that
%% that for some of our test programs jCute did not recognize various paths in our
%% program. 

